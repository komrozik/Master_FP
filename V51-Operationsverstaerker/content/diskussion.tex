\newpage
\section{Diskussion}
\label{sec:diskussion}
    In der Graphen zum Linearverstärker sind eindeutig die zu erwartenden Geraden zu sehen. Das Ausscheren einiger Messwerte ist mit einer Vielzahl von Umgebungseinflüssen zu erklären, wie z.B. nicht beachteter Innenwiderstände mancher Bauelemente.

    Anhand der Bilder vom Oszilloskop ist zu erkennen, dass der Umkehr-Integrator wie erwartet die eingehenden Signale integriert und die entsprechenden Signale ausgibt, die Dreieckspannung wird zur Rechteckspannung usw.
    Für den invertierenden-Differenzierer gilt das gleiche, auch dabei werden die erwarteten Ausgangssignale erkennbar.
    Am besten ist diese Eigenschaft bei dem Ausgangssignal der Rechteckspannung zu sehen, da dabei immer dort spitze Peaks auftreten, wo die Spannung von einem Wert zu anderen springt.

    Die Proportionalität der Ausgangsspannung zur Frequenz ist in den Graphen jedoch nur ansatzweise erkennbar.
    Diese Tendenzen sind auch nur bei tiefen Frequenzen zu erkennen da der Operationsverstärker in seinem Aufbau einem Tiefpass ähnelt.
    Der Trend des Graphen des Integrators zeigt nach unten, was der Proportionalität $\propto 1 / \nu$ entspricht und der Trend des Graphen des Differentiators zeigt nach oben, was wiederrum der Proportionalität $\propto \nu$ darstellt.

    Ein fehlerhafter Operationsverstärker und ein paar andere Faktoren haben den Aufbau des Versuches zeitaufwendig werden lassen, weshalb nur die Aufgaben bis zu den Schmitt-Triggern bearbeitet werden konnten.
