\newpage
\section{Diskussion}
Im dem Versuch wurden die Komponenten eines HeNe-Lasers analysiert. Hierbei wurde zunächst die
Stabilitätsbedingung bei verschiedenen Resonator-Spiegeln-Konfigurationen überprüft. 
Dabei ist deutlich geworden, dass sich für die konkav/konkav Anordnung ($r_1=r_2$) sowie für
die plan/konkav Anordnung eine kritische Stabilität bei einer Resonatorlänge von $1,40\,$m finden lässt.
eine maximale Resonatorlänge konnte sich für die konkav/konkav Anorndung bei $2,80\,$m finden.
Da die Apperatur aufgrund ihrer optischen Schiene auf $2\,$m begrenzt ist, ist diese allerdings von nachrangigier Bedeutung.
Die letzte erfolgreiche Messung konnte dabei bei plan/konkav bis $1,30\,$m durchgeführt werden, da die Justierung mit
steigender Resonatorlänge zunehmend schwieriger wird.\\
Die Messung der verschiedenen Moden deckt sich mit der theoretischen Intensitätsverteilung.
Besonders bei der TEM$_{00}$-Mode folgen die Messungen der vorhergesagten Theoriekurve. Bei der TEM$_{10}$ decken sich Theorie und Messung 
nicht so genau wie bei der TEM$_{10}$-Messung. Dies könnte über weiter Messungen und kleinere Schrittweiten in x-Richtung verbessert werden.
Auch die Reduzierung von Störlicht könnte eine messbare Verbesserung bewirken, da durch die so schon kleinen Interferenzmaxima der TEM$_{10}$-Mode
weiteres Licht die Messung stark beeinflussen.
Allerdings kann man auch hier die Theorie an den Messwerten erkennen.
Die Analyse der Polarisation zeigt wie erwartet eine Periodizität von $\pi$. Die konstante Phase die
durch die Interpolation der Messwerte mit der Theorie ermittelt wurde, folgt dabei aus dem nicht perfekt
parallel ausgerichteten Laser.\\
Bei der Bestimmung der Wellenlänge konnte mithilfe verschiedener Beugungsgitter die Wellenlänge durch eine Mittlung auf
$\bar{\lambda}=651,71\,$nm bestimmt werden. Die theoretische Wellenlänge des Lasers ist mit $\lambda_{\text{Theo}}=632,8\,$nm bekannt.
Somit folgt eine Abweichung von ca. $3\,$\% zum theoretischen Wert und ist somit ein zufriedenstellendes Ergebnis.
Die Abweichung können dabei weiter verkleinert werden, indem weitere Beugungsmaxima weiterer $n$-ter Ordnung für die jeweiligen Beugungsgitter
vermessen werden. Zudem könnte die Ungenauigkeit in der Abstandsmessung durch ein Laser-Abstandsmessgerät stark verringert werden gegenüber der Messung mit Massband und Geodreieck.
\label{sec:Diskussion}