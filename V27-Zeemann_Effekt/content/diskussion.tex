\newpage
\section{Diskussion}
\label{sec:diskussion}
Die beiden Übergänge des $\sigma$ polarisierten Lichtes bei \SI{480}{\nano\metre} und bei \SI{643,8}{\nano\metre} konnten bei der Versuchsdurchführung beobachtet werden.
Der $\pi$ Übergang bei \SI{480}{\nano\metre} konnte nicht beobachtet werden da der vorhandene Magnet mit der vorhandenen Spannungsquelle nicht die entsprechende Magnetfeldstärke erreichen konnte.

In den aufgenommen Bildern kann man die Aufspaltung der Magnetfeldlinien durch den Zeemann effekt sehr gut erkennen, allerdings ist es nicht möglich die Paare den unaufgespalteten Linien immer genau zuzuordnen.
Da mit den Kameraaufnahmen die Positionen der Maxima nicht genau definiert werden können ist dort eine Fehlerquelle beim Auslesen zu erwarten.

Die beiden Landé Faktoren wurden mit relativen Abweichungen von \SI{13+-5}{\percent} und \SI{18+-5}{\percent} bestimmt.
Die Unsicherheit dieser bestimmung kann auf das ungenaue Ablesen und auf die eventuelle ungenaue bestimmung des Magnetfeldes zurückgeführt werden.
\vspace{2cm}

\centering
\begin{tabular}[h]{|l|l|l|l|}
    \hline
    Übergang & $g_{exp}$ & $g_{Lit}$ & relative Abweichung\\
    \hline
    rot & \num{1.13+-0.06} & \num{1} & \SI{13+-5}{\percent}\\
    $\text{blau}_{\sigma}$ & \num{1.00+-0.06} & \num{1.75} & \SI{42+-3}{\percent}\\
    $\text{blau}_{\pi}$ & \num{0.738+-0.023} & \num{0.5} & \SI{47+-4}{\percent}\\
    \hline
\end{tabular}