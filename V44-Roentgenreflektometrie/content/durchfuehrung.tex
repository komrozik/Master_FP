\newpage
\section{Durchführung}
\label{sec:Durchfuehrung}
\begin{description}
    \item[Invertierender-Linearverstärker] Für den invertierenden Linearverstärker wird dieser zunächst mit den Wiederständen $R_1 = \SI{1}{\kilo \ohm}$ und $R_2 = \SI{10}{\kilo \ohm}$ aufgebaut. Anschließend wird für diesen Aufbau eine Sinusspannung angelegt und das Ausgangssignal mit dem Eingangssignal auf einem Oszilloskop untersucht. Mit dem Oszilloskop werden dabei die Verstärkung und die Phasenverschiebung festgestellt.
    \item[Umkehr-Integrator] Der Integrator wird mit den Bauteilen für $R = \SI{10}{\kilo \ohm}$ und $C = \SI{100}{\nano\farad}$ aufgebaut. Anschließend wird mit dem Oszilloskop untersucht ob die Schaltung das Eingangssignal integriert indem nacheinender eine Sinusspannung, eine Rechtecksspannung und eine Dreiecksspannung angelegt wird. Abschließend wird auch hier der Frequenzgang untersucht. 
    \item[Invertierender-Differenzierer] Beim Differenzierer mit $R = \SI{100}{\kilo \ohm}$ und $C = \SI{22}{\nano\farad}$ wird analog zum Invertierer gearbeitet.
    \item[Schmitt-Trigger, Signalgenerator und Sinusgenerator] Die restlichen Schaltungen wurden aufgrund von Zeitproblemen nicht durchgeführt.
\end{description}