\newpage
\section{Durchführung}
\label{sec:Durchfuehrung}
    Der Versuch wird an einem D8-Diffraktometer durchgeführt, welches im Wesentlichen aus drei Teilen besteht.
    Zunächst wird die divergente Röntgenstrahlung an einer Röntgenröhre mit Kupferanode erzeugt und mithilfe eines Göbelspiegels fokussiert und monochromatisiert.
    Anschließend wird die Strahlungsintensität durch einen Autoabsorber angepasst und durch eine Blende wird der Strahl emittiert.
    Die Röntgenröhre ist dabei im Winkel bezüglich dem Probentisch verstellbar.
    Dieser Probentisch ist der zweite Teil des Versuchsaufbaus und lässt sich in x-, y- und z-Richtung verschieben um verschiedene Messungen durchzuführen.
    Den dritten und letzteen Teil des Versuchaufbaus bildet ein Detektor welcher auch im Winkel bezüglich des Probentisches einstellbar ist.
    Mithilfe des Computerprogramms XRD Commander können die Bauteile angesteuert werden und die Messungen aufgenommen werden.

    \subsection{Justage des Diffraktometers}
        Zunächst soll die \SI{0}{\degree} Lage des Detektors bestimmt werden.
        Um den Strahl zu vermessen wird zunächst die Probe in der z-Richtung aus dem Strahlengang zu entfernen und der Detektor und Emitter auf die \SI{0}{\degree} markierung gefahren.
        Anschließend werden Detektor und Emitter in einem kleinen Winkelbereich um \SI{0}{\degree} bewegt um das Maximum zu bestimmen und dieses als neue \SI{0}{\degree} Position festzulegen.