\newpage
\section{Diskussion}
\label{sec:diskussion}
    Um die endgültige Bestimmung der Parameter des benutzten Polysterol-Silizium-Schichtsystems durchführen zu können, mussten die Eigenschaften des verwendeten Röntgenstrahls und der Geometrie des Aufbaus in Betracht gezogen und bestimmt werden.

    Der Geometriewinkel ab dem nur noch ein Teil des Strahls die Probe trifft wurde auf zwei verschiedene Weisen bestimmt.
    Beide Werte liegen nah beieinander und so wird einfach ihr Mittelwert $\overline{\alpha_{\mathrm{geo}}} = 0,6108$ in den weiteren Rechnungen zum Geometriefaktor genutzt. \\
    Im Gegensatz zur Kurve in \autoref{fig:Reflektivitaetskurve1} ist bei der Kurve in \autoref{fig:Reflektivitaetskurve2} ein deutliches Plateau bis zum kritschen Winkel der Totalreflektion zu sehen, wo die Reflektivität gleich eins ist. Das ist ein Indiz dafür, dass die Korrektur der Messwerte mithilfe des Geometriefaktors gelungen ist.

    Anschließend wurde der Parratt-Algorithmus für raue Grenzflächen verwendet, um eine Theoriekurve an die gemessene korrigierte Reflektivitätskurve anzupassen.
    Die Parameter wurden händisch so gewählt, dass die Kiessig-Oszillationen im Bereich $0,4°$ bis $0,9°$ möglichst gut mit der Theoriekurve übereinstimmen.
    Zum Vergleich wurden in demselben ungefähren Bereich die Maxima dazu benutzt die Schichtdicke der Polysterol-Schicht zu bestimmen.
    Die beiden somit erhaltenen Schichtdicken stimmen nicht ganz überein.
    Dies könnte daran liegen, dass bei der Methode mit den Maxima nur eindeutig erkennbaren Maxima benutzt wurden und die Oszillationen in anderen Bereichen etwas andere Abstände besitzen.
    \begin{align*}
        d &= (830 \pm 24) \; \text{\AA} \\
        d_{\mathrm{p}} &= 860 \; \text{\AA}
    \end{align*}
    Eine andere Ursache könnte sein, dass die händische Anpassung der Parameter im Parratt-Algorithmus einfach nicht so gut gelungen ist.
    
    Es folgt eine kurze Beschreibung der groben Veränderungen, die eine Variation der einzelnen Parameter mit sich bringt: \\
    Die Schichtdicke $d_{\mathrm{p}}$ ist für die Stauchung und Streckung entlang der x-Achse zuständig.
    Die Variationen der Dispersion von Polysterol $\delta_2$ und der Rauheit der Grenzfläche zwischen Vakuum und Polysterol $\sigma_1$ bewirken jeweils eine Stauchung und Streckung entlang der y-Achse.
    Die Veränderung der Dispersion von Silizium $\delta_3$ bewirkt eine Verschiebung entlang der x-Achse.
    Mit der Rauheit der Grenzfläche zwischen Polysterol und Silizium $\sigma_2$ wird die Krümmung der Kurve beschrieben.

    Weiterhin werden die mit dem Parratt-Algorithmus erhaltenen Dispersionen für Polysterol und Silizium mit ihren Literaturwerten verglichen.
    \begin{align*}
        \delta_2 &= 1,0 \cdot 10^{-6} & \delta_{\mathrm{PS,lit}} &= 3,5 \cdot 10^{-6} & a_{\delta_2} &\approx 72\;\%\\
        \delta_3 &= 7,0 \cdot 10^{-6} & \delta_{\mathrm{SI,lit}} &= 7,6 \cdot 10^{-6} & a_{\delta_3} &\approx 7,9\;\%
    \end{align*}
    Die daraus berechneten kritischen Winkel weisen ungefähr das gleiche Verhältnis zwischen den relativen Abweichungen zu den Literaturwerten auf:
    \begin{align*}
        \alpha_{\mathrm{c,PS}} &= 0,0810° & \alpha_{\mathrm{c,PS,lit}} &= 0,153° & a_{\alpha_{\mathrm{c,PS}}} &\approx 47\;\%\\
        \alpha_{\mathrm{c,Si}} &= 0,2144° & \alpha_{\mathrm{c,Si,lit}} &= 0,223° & a_{\alpha_{\mathrm{c,Si}}} &\approx 3,7\;\%
    \end{align*}

    Es gibt keine Literaturwerte für die Rauheiten. Da die Disperionen in den richtigen Größenordnungen sind, lässt sich zu den Rauheiten nur sagen, dass die erhaltenen Rauheiten eine grobe Abschätzung der tatsächlichen Rauheiten der beiden Grenzflächen sind.
    \begin{align*}
        \sigma_1 &= 8,0 \cdot 10^{-10} \frac{1}{\text{\AA}}\\[2pt]
        \sigma_2 &= 6,8 \cdot 10^{-10} \frac{1}{\text{\AA}}
    \end{align*}

