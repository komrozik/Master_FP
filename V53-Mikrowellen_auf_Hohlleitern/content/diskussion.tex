\newpage
\section{Diskussion}
\label{sec:diskussion}
    \subsubsection*{Untersuchung dreier Moden auf dem Oszilloskop}
        Im ersten Teil des Versuches wird der Zusammenhang zwischen Reflektorspannung und Resonanzfrequenz untersucht.
        Es bilden sich die erwarteten Peaks bei bestimmten Reflektorspannungen, was die theoretischen Annahmen bestätigt. \\
        Die relativ großen Unsicherheiten der Abstimm-Empfindlichkeiten sind durch die groß gewählten systematischen Messfehler bedingt.
        Dies liegt wiederum daran, dass die Delle, mit deren Hilfe die Reflektorspannung und Frequenz abgelesen wurde, bei halber Leistung an den steilen Flanken schlecht im Oszilloskop zu erkennen war.
        Außerdem sind die Kurven etwas asymmetrisch, was jedoch keinen Widerspruch zur Theorie bildet, sondern auf einen systematischen Fehler bei der Durchführung der Messung schließen lässt.

        Eine weitere Bestätigung der Genauigkeit der Messung liefert die Bestimmung der Modenzahlen der drei Peaks. Die Abweichung ist minimal sodass eine eindeutige Identifizierung der Moden mit den Modenzahlen $n = 5, 6, 7$ möglich ist.

    \subsubsection*{Untersuchung der Frequenz}
        Der zweite Teil des Versuches beschäftigte sich mit einer genaueren Messung der Frequenz.
        Diese wird zuerst mithilfe des SWR-Meters und dann auf eine zweite Art mithilfe der Messung der Wellenlänge im Hohleiter ermittelt.
        \begin{align*}
            f = (8996 \pm 0,5) \; \mathrm{MHz} \\
            f_{\lambda} = (8983 \pm 20) \; \mathrm{MHz}
        \end{align*}
        Die Werte sind leicht verschieden, dennoch überschneiden sich ihre Fehlerintervalle.
        Bei der Messung mit dem SWR-Meter gab es wenig Gelegenheit für einen systematischen Fehler.
        Während der Durchführung der Berechnungen für die zweite Methode ist jedoch aufgefallen, dass die Frequenz sehr sensitiv gegenüber einer leicht veränderten Wellenlänge ist. Wäre nur der Abstand ersten zwei Maxima verwendet worden so würde $f_{\lambda} \approx 9000\;$MHz sein. Das heißt, dass die mit der zweiten Methode bestimmte Frequenz etwas anders gewesen wäre, wenn über mehrere Maxima gemittelt worden wäre.

        Trotzdem ergeben sich sehr ähnliche Phasengeschwindigkeiten für die beiden Frequenzen.
        Diese sind ca. anderthalb Mal so groß wie die Lichtgeschwindigkeit, was jedoch keinen Widerspruch zur Relativitätstheorie darstellt.
        Denn nur die Gruppengeschwindigkeit, die Geschwindigkeit der Wellenpakete in denen die Informationen enthalten sind, muss kleiner als die Lichtgeschwindigkeit sein.
        In dispersiven Medien laufen Wellenpackete auseinander.
        Aus diesem Grund kann ein Punkt gleicher Phase relativ zum Wellenpacket schneller und so die Phasengeschwindigkeit höher als die Lichtgeschwindigkeit sein.

    \subsubsection*{Dämpfungskurve}
        Die aufgenommene Dämpfungskurve entspricht sehr gut der abgelesenen Eichkurve. Die konstante Verschiebung der beiden Kurven bei allen Datenpunkten außer bei einer Mikrometereinstellung von 0, deutet auf einen systematischen Messfehler in Form einer nicht geeichten Skala hin. Eine weitere mögliche Erklärung wäre ein nicht berücksichtigter Offset im SWR-Meter.
    
    \subsubsection*{Bestimmung des Stehwellenverhältnisses}
        Die mit der direkten Methode erhaltenen Werte für das SWR können nicht einezeln bewertet werden, da keine Litaratur- bzw. Vergleichswerte vorliegen.
        Die Theorie wird jedoch darin, dass das SWR bei steigender Reflektion an der Sonde steigt.
        Die Messunsicherheit steigt dabei mit größeren SWR, da diese durch die wechselnde Skala des SWR-Meters gegeben ist.

        Die mit der 3 dB-Methode und der Abschwächermethode bestimmten SWR-Werte bei einer Eindringtiefe von \SI{9}{mm} können verglichen werden.
        \begin{align*}
            S_{\mathrm{3dB}} &= 9.23 \pm 0.38 \\
            S_{\mathrm{schw}} &= 12,59 \pm 2,05
        \end{align*}
        Die Fehlerintervalle überschneiden sich knapp nicht.
        Das könnte daran liegen, dass einer der Werte für die Dämpfung schlecht abgelesen wurde.

    
    Nach Abschluss dieses Versuchs kann gesagt werden, dass sein Ziel, nämlich die größere Vertrautheit mit Mikrowellen und stehenden Wellen, durch die vielen verschiedenen Messungen erreicht wurde.
